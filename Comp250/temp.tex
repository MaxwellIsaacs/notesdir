% Created 2024-11-19 Tue 18:15
% Intended LaTeX compiler: pdflatex
\documentclass[11pt]{article}
\usepackage[utf8]{inputenc}
\usepackage[T1]{fontenc}
\usepackage{graphicx}
\usepackage{longtable}
\usepackage{wrapfig}
\usepackage{rotating}
\usepackage[normalem]{ulem}
\usepackage{amsmath}
\usepackage{amssymb}
\usepackage{capt-of}
\usepackage{hyperref}
\date{\today}
\title{Midterm II, Review II: The Real Deal}
\hypersetup{
 pdfauthor={},
 pdftitle={Midterm II, Review II: The Real Deal},
 pdfkeywords={},
 pdfsubject={},
 pdfcreator={Emacs 29.4 (Org mode 9.7.11)}, 
 pdflang={English}}
\begin{document}

\maketitle
\tableofcontents

\section{Question 1 almost, basically check}
\label{sec:orgf3d8428}

\begin{enumerate}
\item o(n+m)
\item o(n\textsuperscript{2})
\item o(n * m)
\item o(n\textsuperscript{m})

why c and d are wrong.
c is close, but because of i < m, the notation is actually o (min(n,m) * m)
d is wrong completely, but I changed it. Basically, the second iteration is exponential, and is multiplied as many times as m. so n is multiplied by itself n\textsuperscript{m} times, and this is how many loops you do (as seen in k)
\end{enumerate}
\section{Question 2 nah}
\label{sec:org550183f}

f(n) = n\textsuperscript{2}
g(n) = n
h(n) = logn


first statement: true
 second statement: true
 third statement: false
 why? because lim \(\lim_{n \to \infty} \frac{g(n)}{h(n)} = \infty\). they do not change at the same rate, this statement is wrong and stupid. what retard wrote this?
\section{Question 3 check?}
\label{sec:orge7ca13a}
\begin{enumerate}
\item o(1)
\item o(n)
\end{enumerate}
\section{Question 4 check}
\label{sec:org2fc1d17}
D R D R D [ l d ] [d]

d r d r d [r d] [d] [ld]
\section{Question 5 check}
\label{sec:orgc320f9a}
\begin{itemize}
\item {[}11 21 \_36\_]
\item {[}11 57]
\end{itemize}

{[}57 11 25]
{[}25 (57 * 11)]
\section{Question 6}
\label{sec:org622d288}
\begin{itemize}
\item {[}- p f -]
\item {[}- p F A]
\item {[}- - F A]
\item {[}G - F A]
\end{itemize}
\section{Question 7}
\label{sec:org1e5b261}
reverse a queue \textbf{q} up until \textbf{k} elements
\section{Question 8}
\label{sec:orgde43916}
Blank One: !d1.isEmpty();
Blank two: d2.addFirst (d1.removeLast());
\section{Question 9}
\label{sec:orgf807584}
Both train() and pat(), since interface functions need to be defined
\section{Question 10}
\label{sec:org12f1254}
\begin{enumerate}
\item no error
\item legal
\item illegal
\item illegal, you cannot directly initialize an interface
\end{enumerate}
\section{Question 11}
\label{sec:org71d6bb8}
Just iterable and comparable
\section{Question 12}
\label{sec:orgf925e83}
E\^{}+ = \{2 + E\^{}+(n+1)\}
\section{Question 13}
\label{sec:org0e8155a}
\[
\sum_{i=1}^{n} (2i - 1)
\]

\[
\sum_{i=1}^{1} (2i - 1)
\]

\begin{itemize}
\item this is equal to 1, as is \textbf{n\textsuperscript{2}}
\end{itemize}

\[
\sum_{i=1}^{n+1} (2i - 1)
\]

\begin{itemize}
\item This is equal to 4, as is n\textsuperscript{2} \(\therefore\) the statement is true by proof of induction
\end{itemize}
\end{document}
